% Please do not change the document class
\documentclass{scrartcl}

% Please do not change these packages
\usepackage[hidelinks]{hyperref}
\usepackage[none]{hyphenat}
\usepackage{setspace}
\doublespace

% You may add additional packages here
\usepackage{amsmath}

% Please include a clear, concise, and descriptive title
\title{Can the agile principles be useful tools for university lecturers to facilitate learning in their courses?}

% Please do not change the subtitle
\subtitle{COMP150 - Agile Development Practice}

% Please put your student number in the author field
\author{1605629}

\begin{document}

\maketitle

\abstract{Please include an abstract of at most 100 words (these do not count towards your word count).}

\section{Introduction}

In the modern higher educational world, we are coming across a much higher level of diversity in the preparedness of students taking many of the courses at universities. This increase in diversity is due to the number of students taking courses that they may not have taken suitable A-level qualifications for. Many tutors can see this as an impediment to teaching these students, but I believe that it necessitates a change in the way teaching is performed at the tutor level. The method I feel is most effective in this modern teaching setting is a combination of the Agile Principles with the teacher taking up the role of Scrum Master. In the rest of this paper I aim to show the benefits of teaching in such a way at the higher education level.

\section{The ‘Agile’ way of Teaching}

The first and most important principle that can be included in teaching is the principle of Cycles. To properly incorporate Agile into the way a tutor teaches they must ask themselves 6 questions [1].
They must first ask themselves “What is the Bottleneck to learning in the class?” This would be the place in which most students will become stuck or find hard to understand. [1] After they have located this bottleneck in their class, they must then ask themselves “How would an Expert complete the task identified in the bottleneck?” This helps them to understand what they hope to teach their students in the long term. They must then ask “How can the expert’s tasks be modelled in class?”. This helps the tutor to understand how they can come to teaching their students the task in a better way. The next key question is “How can the students actively practice these tasks?”. This can be through the modelling of assignments that can be completed independently by the students with guidance from the tutor. They must then get quick feedback from the tutor before the next assignment. Each assignment should be a short task that can be completed relatively quickly, so the tutors feedback can be implement iteratively into the student’s assignments. Within this they must then ask how they can motivate their students. This requires making the assignments relevant and enjoyable. The last question that must be asked is “How well are the students mastering these learning tasks?”. This is a critical part of the process as it forms the ‘review’ section of the cycle. This means that during this section, tutors need to have direct contact with each student and have a review session with them to find out personally how they have have managed, and if there are any blockers to their learning. After this stage they go back to the mini assignments once the blockers have been removed and attempt them again.

\section{The Tutor as the Scrum Master}

The most pivotal role during any form of higher education is the tutor, and as result of this, they should take the most critical role within the Agile principles; that of the scrum master. In taking up this role they oversee the daily stand-ups with their students that are a crucial part of the agile principles. During these Stand-ups, the tutor will have several very important roles to perform that will overcome problems that, with the current method of teaching, can be very hard to overcome. 

\section{Conclusion}

Write your conclusion here. The conclusion should do more than summarise the essay, making clear the contribution of the work and highlighting key points, limitations, and outstanding questions. It should not introduce any new content or information.

\bibliographystyle{ieeetran}
\bibliography{references}

\end{document}